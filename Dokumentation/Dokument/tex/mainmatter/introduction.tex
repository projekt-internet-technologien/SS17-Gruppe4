\chapter{Aufgabenstellung}
\label{cha:intro}

Das Projekt Internet Technologie umfasst die praktische Anwendung der in der zugehörigen Vorlesung vorgestellten Konzepte und Systeme. Im Speziellen ist das Projekt auf die zunehmende Verbreitung und Integration digitaler Services in die alltägliche Umgebung ausgelegt. Zusammengefasst unter Schlagworten wie Internet of Things oder Ambient Computing werden dabei digitale Rechenwerke, Aktorik und Sensorik miteinander vernetzt um bestehende Funktionalitäten zu erweitern, beispielsweise einen Lichtschalter zu einer Lichtsteuerung zu machen oder gänzlich neue Möglichkeiten zu Interaktion mit der Umgebung zu schaffen. \\

Vor diesem Hintergrund sollen die teilnehmenden Gruppen in Kollektivarbeit eigene Konzepte entwickeln, welche sich zu einem intelligenten Raum zusammenfassen lassen. Das Projekt nutzt Raspberry Pi und Arduino als Entwicklungsplattform und bedient sich zeitgemäßer Technologien wie CoAP, SPARQL oder WebSockets. Die einzelnen Projekte werden mit Maven gruppenübergreifend standardisiert, die Versionsverwaltung erfolgt mit Git.

Ein weiteres Ziel ist die Erstellung einer Bibliothek für den Zugriff auf die verwendeten Hardwarekomponenten.