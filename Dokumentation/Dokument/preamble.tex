%%%%% General packages & Utilities

% Allow for German "Umlaute"
% Note that for this to work, all files containing Umlaute must be saved in the UTF-8 file format!
\usepackage[T1]{fontenc}
\usepackage[utf8]{inputenc}
\usepackage[hyphens]{url}
% Tables
\usepackage{booktabs}
\usepackage[table,xcdraw]{xcolor}
\usepackage{multirow}

\usepackage{subfigure}
% Code
\usepackage{listings}
% colordefinitions
\definecolor{lightlightgray}{rgb}{0.01,0.199,0.1}
\definecolor{light-gray}{gray}{0.9}
\definecolor{dark-gray}{gray}{0.45}

\lstset{
  breaklines=true,
  basicstyle=\ttfamily,
  backgroundcolor=\color{light-gray},
  showstringspaces=false,
  frame=ltrb,
  language=Matlab,
  tabsize=2,
  numbers=left,
  numberstyle=\small,
  numbersep=8pt,
  language=Java,
  morekeywords={*, factorial, sum, erlang},
  keywordstyle=\color{orange}\textbf,
  commentstyle=\color{green}\textit,
  stringstyle=\color{blue}
}

%\lstset{
%numbers=left, 
%numberstyle=\small, 
%numbersep=8pt, 
%frame = single, 
%language=Java,
%breaklines=true,
%framexleftmargin=15pt}

% Make todo notes in the thesis.
% Simply change the status to "final" to suppress printing of the notes and their list.
\usepackage[status=draft]{fixme}

% Useful in defining custom shortcut commands
\usepackage{xspace}

% Load english and german language hyphenation and options.
% The language put last here is selected as the main language for the document,
% specifying keywords such as Table of Contents / Inhaltsverzeichnis, etc.
\usepackage[ngerman, english]{babel}
\usepackage{csquotes}


%%%%% Style

% Font type
\usepackage{lmodern}

% Line spacing
\usepackage{setspace}
%\singlespacing
\onehalfspacing
%\doublespacing



%%%%% Graphics packages

\usepackage{graphicx}

%% add subfolder containing figures to the graphics path
\graphicspath{{figures/}}

% Subfigures
\usepackage{caption}
%\usepackage{subcaption}

% Powerful package for programmatically producing nice vector graphics
\usepackage{tikz}


% Package for easily drawing electrical circuits (uses tikz)
\usepackage[european]{circuitikz}

% Package for creating plots in latex using tikz internally
%\usepackage{pgfplots}
%\pgfplotsset{compat=1.11}
%\usepgfplotslibrary{units}


\newcommand*{\quelle}{%
  \footnotesize Quelle:
}

%%%%% Referencing, Glossary, Bibliography

% Allow for cross-referencing other parts of this documents
% -- hidelinks prevents ugly red boxes around clickable items. Alternatively, these could be styled differently.
% -- pdfusetitle adds author and title information to meta data
\usepackage[hidelinks, pdfusetitle]{hyperref}

% Easier cross-referencing, using the \cref and \Cref commands
\usepackage{cleveref}

% Glossaries must be loaded after hyperref
\usepackage[style = long, nolist, acronym, nonumberlist, nopostdot, toc]{glossaries}
\newglossary{symbols}{sym}{sbl}{List of Mathematical Symbols}
%\makenoidxglossaries

% Bibliography
\usepackage[backend=biber, style=ieee, sorting=nyt]{biblatex}
% Change the separator between title and subtitle of a book from comma to colon
\renewcommand{\subtitlepunct}{\addcolon\addspace}
% Make the space between author names and actual reference in the \textcite
% command non-breakable.
%\renewcommand\namelabeldelim{\addnbspace}
\addbibresource{references.bib}

% Add lines to table of contents for list of figures, tables, etc. and for bibliography.
% Alternatively (e.g. when not using a KOMA document class), this could be achieved by 
% adding
%     \addcontentsline{toc}{chapter}{List of Figures}
% before \listoffigures, and similarly for the others.
\KOMAoptions{
  listof=totoc,
  bibliography=totoc
}



%%%%% Maths and computer science packages

\usepackage{amsmath}
\usepackage{amsthm}
\usepackage{amsfonts}
\usepackage{mathtools}

% introduce new type of theorems (uses amsthm package)
\newtheorem{lemma}{Lemma}

% Improvement of \left and \right due to 
% http://tex.stackexchange.com/questions/2607/spacing-around-left-and-right
\usepackage{mleftright}

% Produce pseudo code sections
% See http://tex.stackexchange.com/a/230789/64293 for a discussion of the various
% packages available for this task.
\usepackage{algorithm}
\usepackage{algpseudocode}

% Correctly display SI units
\usepackage{siunitx}



\makeatletter
\newcommand{\addloflink}[1]{% \addloflink{<URL>}
  \addtocontents{lof}{\begingroup\def\protect\@dotsep{10000}% Remove dots in LoF for this entry
    \protect\contentsline{figlink}{\protect\numberline{}\url{#1}}{}{}%
    \endgroup}% Restore dots in LoF for future entries
}
\newcommand{\l@figlink}{\@dottedtocline{1}{1.5em}{2.3em}}
\makeatother


%%%%% Abbreviations and macros

% Order must be first macros then glossary, since the latter uses the former

\newcommand*\submitdate{1. April 2020}


%% ABBREVIATIONS & MACROS

\newcommand{\swname}[1]{\texttt{#1}}


%% GENERAL MATHS STUFF

\newcommand*{\diff}{\mathrm{d}}
\newcommand*{\pdiff}[0]{\partial}
\newcommand*{\dd}[2][]{\frac{\pdiff #1}{\pdiff #2}}


%% CHAPTER MATERIALS AND METHODS

\newcommand*{\vel}{v}
\newcommand*{\loc}{x}
\newcommand*{\acc}{a}
\newcommand*{\mass}{m}
\newcommand*{\force}{f}


%% CHAPTER CONCEPT

\newcommand*{\Bfun}{B}
\newcommand*{\Afun}{A}




%%%%% Emacs-related stuff
%%% Local Variables: 
%%% mode: latex
%%% TeX-master: "main"
%%% End: 

\loadglsentries{glossary}



%%%%% Emacs-related stuff
%%% Local Variables: 
%%% mode: latex
%%% TeX-master: "main"
%%% End: 
